\documentclass{scrartcl}

\usepackage[hidelinks]{hyperref}
\usepackage[none]{hyphenat}

\title{Research Journal}
\subtitle{COMP130 - Research Journal}

\author{KG197307}

\begin{document}

\maketitle

\section*{Topic}

This paper will cover a summary of multiple papers covering the topic of the use of games for education.

\section*{Introduction}

In the past 10 years paper 3 states the "Game industry as a whole has seen massive growth, even in the academic spotlight" [1]\cite{}. The younger generation have grown up around technology such as laptops and mobile phones and have experience with technology from a young age. This paper aims to cover how computer learning through "serious games"[1][2][3][4][6] and simulations are becoming a more popular way to educate people of all ages for different purposes including skilled jobs and degrees. 

\section*{Research}

The majority of students today that have grown up around a lot of technology so naturally they take much more of an interest in learning that is computer based as it is more engaging for them. This subject has gained a lot of interest over the past 5 years with an increasing number of research papers being published in the matter. Of these papers this technique is used to educate young and old people for many different purposes. 

In paper 1 i found these 'serious games' are being used to train people in emergency situations and even teamwork is added to this aspects with multiple people being trained at once. These students are so emerged in the game it is "Better support for real life experiences" [1] states 'Using games for teaching crisis communication in higher education and training'. 

Current technology allows us to make custom controllers to simulate whatever is needed. Even surgeons are being trained via computer simulations currently and are finding great success that they have whole training suites to educate students. http://ieeexplore.ieee.org.ezproxy.falmouth.ac.uk/mediastore/IEEE/content/media/7556943/7561497/7561511/7561511-fig-14-large.gif

I also came across many papers mentioning that their simulation software/game was made in Unreal Engine which such a widely used engine for the games industry and available to a lot of people rather than having custom built engines. For example, for UAV simulation training they are using Unreal engine. Although they do state that it needs improvement, overall the learning is there for new people in the job sector. 


Simulation software for job training using VR this time, this is for nursing degree 
People can use this to train at home if they are unable to commit to a training centre 
Again concludes with further development would improve this, what relates all the papers 



\section*{Conclusion}
To summarise these papers I found that many focus on the future by telling us that more development is needed for the software. However all of the research conclude that the use of 'serious games' is efficient at its intended purpose of educating people in the desired ways. From the research that I have read I can see this this sector of the games industry expanding a lot further since it helps people in so many ways. 

\bibliographystyle{ieeetran}
\bibliography{initial_references}

[1] Ki, L., Holen, S., Vold, T., Venemyr, G.O. and Braun, R., 2016, September. Using games for teaching crisis communication in higher education and training. In Information Technology Based Higher Education and Training (ITHET), 2016 15th International Conference on (pp. 1-6). IEEE.

[2] Morsi, R. and Mull, S., 2015, October. Digital Lockdown: A 3D adventure game for engineering education. In Frontiers in Education Conference (FIE), 2015. 32614 2015. IEEE (pp. 1-4). IEEE.

[3] A. Marquez-Tellez, M. Vazquez-Briseno, J. I. Nieto-Hipolito and J. D. S. Lopez, "Using sensor fusion in a serious game for children nutrition education," 2017 International Conference on Electronics, Communications and Computers (CONIELECOMP), Cholula, Mexico, 2017, pp. 1-6.


[4] Song, K., Xu, H., Ding, Y. and Li, H., 2016, November. A target tracking realization method of UAV simulation training system. In Control, Automation, Robotics and Vision (ICARCV), 2016 14th International Conference on (pp. 1-4). IEEE.

[5] Elliman, J., Loizou, M. and Loizides, F., 2016, September. Virtual Reality Simulation Training for Student Nurse Education. In Games and Virtual Worlds for Serious Applications (VS-Games), 2016 8th International Conference on (pp. 1-2). IEEE.

[6] Ivaschenko, A., Gorbachenko, N., Kolsanov, A. and Kuzmin, A., 2016, April. Surgery scene representation in 3D simulation training SDK. In Open Innovations Association and Seminar on Information Security and Protection of Information Technology (FRUCT-ISPIT), 2016 18th Conference of (pp. 75-84). IEEE.

\end {document}
